\documentclass{simplenotes}

\title{Ejemplo de \texttt{simplenotes}}
\author{Jeremías Martinez Ciucio}
\date{\today}

\begin{document}

\maketitle
\newpage

\section{Introducción}

Esta es una demostración de la clase \texttt{simplenotes}. Incluye teoremas, definiciones, notas, cajas con íconos y fragmentos de código.

Está basada en la clase \texttt{article}, e incluye ya configuraciones de idioma (con español preconfigurado), fuente, geometría, etc.
En el caso de necesitar artículo, se puede cambiar la línea \texttt{LoadClass} para que se base en \texttt{report}


% ===================================


\section{Entornos tipo teorema} \label{sec:teoremas}

Son los entornos \texttt{theorem}, \texttt{definition} y \texttt{lemma}. Estos pueden estar numerados o no.


\subsection{Entornos sin numerar}
Se usan de la siguiente forma:
\begin{verbatim}
  \begin{<entorno>}{<título>}
    ...
  \end{<entorno>}
\end{verbatim}
Donde \texttt{<entorno>} es el tipo de entorno. Y de forma opcional, \texttt{<título>} es el título que aparecerá encima. 

\noindent
\textbf{Ejemplo:}
\begin{theorem*}{Teorema de Pitágoras}
En un triángulo rectángulo, el cuadrado de la hipotenusa es igual a la suma de los cuadrados de los catetos.  
Es decir, si el triángulo tiene catetos $a$ y $b$, y hipotenusa $c$, entonces:
\[
c^2 = a^2 + b^2.
\]
\end{theorem*}


\subsection{Entornos numerados} 
Estos, a diferencia de los otros, son numerados automáticamente por \LaTeX.
Como ventaja adicional permiten referenciarlos en el texto.

Su uso es el siguiente:
\begin{verbatim}
\begin{<entorno>}{<título>}{<etiqueta>}
  ...
\end{<entorno>}
\end{verbatim}
Donde \texttt{<entorno>} es el tipo de entorno, \texttt{<título>} el título que se muestra en la parte superior, y \texttt{<etiqueta>} la etiqueta que luego se va a utilizar para referenciarlo mediante \verb|\ref{<entorno>:<etiqueta>}|. El siguiente es el lema \ref{lemma:lema-ejem}
\noindent

\textbf{Ejemplo:}
\begin{lemma}{Números pares}{lema-ejem}
Todo número par es divisible por 2.
\end{lemma}


% ===================================


\section{Entornos tipo propiedad rápida}
\noindent
Estos son \texttt{property}, \texttt{proposition}, \texttt{note} y \texttt{convention}. Se utilizan como cualquier entorno normal de \LaTeX.
\begin{property}
La multiplicación por cero anula cualquier número: $a\cdot 0 = 0$.
\end{property}

\begin{proposition}
Si $x > 0$, entonces $x^2 > 0$.
\end{proposition}

\begin{note}
Esta es una nota sobre lo anterior.
\end{note}

\begin{convention}
Se usa la notación $f: A \to B$ para funciones.
\end{convention}


% ===================================


\section{Cajas con íconos}
\noindent
Son los entornos \texttt{observation}, \texttt{remark}, \texttt{question}, \texttt{exercise} e \texttt{important}. Al igual que los de propiedad rápida, se usan como cualquier entorno normal.


\begin{observation}
Observación importante sobre el contenido.
\end{observation}

\begin{remark}
Este es un comentario adicional.
\end{remark}

\begin{question}
Pregunta para reflexionar.
\end{question}

\begin{exercise}
Ejercicio de práctica: Demuestra el teorema anterior.
\end{exercise}

\begin{important}
¡Atención! Este concepto es crítico.
\end{important}


% ===================================


\section{Fragmentos de código}
Podemos insertar código de dos formas distintas. La primera simplemente muestra el código con resaltado por color y números de línea. Para eso utilizamos el entorno \texttt{snippet}.
Para utilizarlo, debemos especificar el lenguaje de programación que contiene, lo que se utiliza para el resaltado de las palabras clave
El siguiente ejemplo usa \verb|\begin{snippet}{python} ... \end{snippet}|

\begin{snippet}{python}
for i in range(5):
    print(i)
\end{snippet}

También podemos envolver el bloque de código en una caja similar a la de los teoremas. Para esto utilizamos el entorno \texttt{codebox}. Este puede ser tanto numerado como no numerado, y sus parámetros funcionan igual que los de los entornos de teorema (ver la sección \ref{sec:teoremas} para su uso).

Es importante que dentro del \texttt{codebox}, utilicemos un \texttt{snippet}, ya que es lo que nos dará el formato del código.

\begin{codebox*}{Código en Python}
  \begin{snippet}{python}
  def saludar(nombre):
      print(f"Hola, {nombre}!")
  \end{snippet}
\end{codebox*}
\noindent
Finalmente, podemos importar archivos externos como un snippet con el comando
\begin{verbatim}
inputsnippet{<archivo>}{<lenguaje>}
\end{verbatim}
de esta forma podemos tener el código en un archivo distinto, lo que mejora la mantenibilidad

\end{document}

